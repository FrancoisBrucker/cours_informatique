\documentclass{article}
\usepackage[french]{babel}
\usepackage[T1]{fontenc}
\usepackage[utf8]{inputenc}
\usepackage{graphicx} % Required for inserting images
\usepackage{listingsutf8}
\usepackage{amsfonts}
\lstset{%
    inputencoding=utf8, 
    literate=
        {é}{{\'e}}{1}%
        {è}{{\`e}}{1}%
        {à}{{\`a}}{1}%
        {â}{{\^a}}{1}%
        {ç}{{\c{c}}}{1}%
        {œ}{{\oe}}{1}%
        {ù}{{\`u}}{1}%
        {É}{{\'E}}{1}%
        {È}{{\`E}}{1}%
        {À}{{\`A}}{1}%
        {Ç}{{\c{C}}}{1}%
        {Œ}{{\OE}}{1}%
        {Ê}{{\^E}}{1}%
        {ê}{{\^e}}{1}%
        {î}{{\^i}}{1}%
        {ï}{{\"i}}{1}%
        {ô}{{\^o}}{1}%
        {û}{{\^u}}{1}%
}
\usepackage{fullpage}
\usepackage[margin=2.1cm]{geometry}
\usepackage{setspace}
\usepackage{todonotes}
\usepackage{amsthm}

\newtheoremstyle{exostyle}% style name
{10pt}% above space
{10pt}% below space
{}% body font
{}% indent amount
{\scshape\bfseries\large}% head font
{\hfill\vspace{5pt}\newline}% post head punctuation
{0pt}% Space after theorem head
{\hfill\thmname{#1}\thmnumber{ #2} -- \thmnote{ #3}}% head spec

\newtheoremstyle{partiestyle}% style name
{1em}% above space
{1em}% below space
{}% body font
{}% indent amount
{\bfseries}% head font
{\vspace{.5em}\newline}% post head punctuation
{0em}% Space after theorem head
{\thmnumber{#2} \thmnote{ #3}}% head spec

\newtheoremstyle{questionstyle}% style name
{.5em}% above space
{.5em}% below space
{}% body font
{}% indent amount
{\bfseries}% head font
{}% post head punctuation
{0em}% Space after theorem head
{Question \thmnumber{#2 }}% head spec

\theoremstyle{exostyle}
\newtheorem{exo}{Exercice}

\theoremstyle{partiestyle}
\newtheorem{partie}{}[exo]

\theoremstyle{questionstyle}
\newtheorem{question}{Question}[exo]
\newtheorem{questionpartie}{Question}[partie]

\title{Examen Terminal UE Algorithmes}
\author{L1 MPCI}
\date{28 mai 2025 - Durée: 2h}

\begin{document}

\maketitle

\begin{center}
{\em\bf Lorsque l'on vous demande d'écrire de décrire ou de donner un algorithme cela signifiera toujours en donner un pseudo-code, justifier de son \underline{exactitude} et de sa \underline{complexité} (en utilisant des $\mathcal{O}$)}

~\\

{\em On rappelle qu'aucun document, ni équipement électrique ou électronique n'est autorisé.

 {\bf Cependant} l'usage d'un coupe-bordures sans fil est toléré.}
\end{center}


\vspace*{1cm}

\paragraph{Les exercices} de cet examen : 
\begin{itemize}
\item sont au nombre de 3;
\item ont tous le même but afficher à l'écran chaque élément de $\mathcal{T}_n$ une fois.
\item sont indépendants (à part le problème à résoudre qui est le même);
\item sont de difficulté {\it a priori} croissante;
\item leur début est ({\it a priori}) plus facile que leur fin.
\end{itemize}

\paragraph{}{L'examen est {\bf long} et ce qui semble simple pour l'examinateur ne l'est pas forcément pour l'étudiant et réciproquement. Il pourra être utile de changer d'exercice plutôt que de rester bloqué sur une question, quitte à y revenir plus tard.}

\paragraph{}{\sc Rendez \underline{des copies séparées pour chaque exercice}, ceci vous permettra de d'y revenir au cours de l'examen sans perdre le correcteur.}

\vspace*{1cm}
Notations :
\begin{itemize}
    \item On note $\mathcal{T}_n$ l'ensemble de toutes {\bf les permutations} du tableau de taille $n$ contenant les entiers allant de $0$ à $n-1$. Par exemple $\mathcal{T}_3 = \{[0, 1, 2], [0, 2, 1], [1, 0, 2], [1, 2, 0], [2, 0, 1], [2, 1, 0]\}$;
    \item Pour un tableau $T$ de taille $n$, on notera $T[\;:k]$ le tableau formé {\bf des $k$ premiers éléments de $T$} (allant des indices 0 à $k-1$). Si $T=[1, 3, 2, 0, 4]$, alors $T[\;:3] = [1, 3, 2]$;
    \item Pour un tableau $T$ $n$, on notera $T[k:\;]$ le tableau formé {\bf des $n-k$ derniers éléments de $T$} (allant des indices $k$ à $n-1$). Si $T=[1, 3, 2, 0, 4]$, alors $T[\;3:\;] = [0, 4]$;
    \item Pour deux tableaux $T$ et $T'$, on notera $T + T'$ le tableau formé de {\bf la concaténation} de $T$ et $T'$. On a $[0, 4] + [1, 3, 2] = [0, 4, 1, 3, 2]$;
    \item Pour un tableau $T$ de taille $n$ {\bf une permutation circulaire de $k$ éléments de $T$} est le tableau $T_k = T[k:\;] + T[\;:k]$. Le tableau $[3, 4, 0, 1, 2]$ est la permutation circulaire de 2 éléments du tableau $[0, 1, 2, 3, 4]$.
    \item Pour deux tableaux d'entiers $T$ et $T'$, $T < T'$ selon {\bf l'ordre lexicographique} si $T \neq T'$ et $T[i] < T'[i]$ pour $i$ le plus petit indice tel que $T[i] \neq T'[i]$;
\end{itemize}

\clearpage

\begin{exo}[Itératif]
    On va utiliser l'ordre lexicographique entre tableaux d'entiers pour générer tous les éléments de $\mathcal{T}_n$. 
    \begin{partie}[Ordre]
        \begin{questionpartie}
            Si $T = [0, 3, 4, 2, 1]$ et $T' = [0, 3, 2, 1 ,4]$. Justifiez pourquoi $T' < T$.
        \end{questionpartie}
        \begin{questionpartie}
            Quels sont le plus petit et le plus grand élément de $\mathcal{T}_n$ ?
        \end{questionpartie}
        \begin{questionpartie}
            Écrivez un algorithme de complexité optimale (vous le justifierez) qui prend en entrée deux éléments $T1$ et $T2$ de $\mathcal{T}_n$ et rend \verb|Vrai| si $T1$ est strictement plus petit que $T2$ et \verb|Faux| sinon. Il sera de signature~:
              \verb|plus_petit(T1: [entier], T2: [entier]) → booléen| 
        \end{questionpartie}
    \end{partie}
    \begin{partie}[Indice $i^\star_{T}$]
        On note $i^\star_{T}\geq 0$ pour $T$ de $\mathcal{T}_n$ un indice $i^\star_{T}\geq 0$ tel que :
        \begin{itemize}
            \item $T[i^\star_{T}:\;]$ est strictement décroissante,
            \item soit $i^\star_{T}=0$ soit $T[i^\star_{T}-1] < T[i^\star_{T}]$.
        \end{itemize}
        \begin{questionpartie}
            Donnez $i^\star_{T}$ et $i^\star_{T'}$ pour $T = [0, 3, 4, 2, 1]$ et $T' = [0, 3, 2, 1 ,4]$.
        \end{questionpartie}
        \begin{questionpartie}
            Démontrez que pour tout élément $T$ de $\mathcal{T}_n$ $i^\star_{T}$ existe et est unique.
        \end{questionpartie}
        \begin{questionpartie}
            Écrivez un algorithme de complexité optimale (vous le justifierez) qui prend en entrée un élément $T$ de $\mathcal{T}_n$ et rend $i^\star_{T}$. Il sera de signature~:
              \verb|i_star(T: [entier]) → entier| 
        \end{questionpartie}
        \begin{questionpartie}
            Démontrez que si $T[\;:i^\star_{T}] = U[\;:i^\star_{T}]$ pour deux éléments $T$ et $U$ de $\mathcal{T}_n$, alors $T \geq U$.
        \end{questionpartie}
        \begin{questionpartie}
            Soit $T \in \mathcal{T}_n$ et $i^\star_{T}$ son indice associé. Montrez que pour tout $U$ de $\mathcal{T}_n$ tel 
            que $T[\;:i^\star_{T}] = U[\;:i^\star_{T}]$ (les $i^\star_{T}$ premiers éléments sont identiques pour $T$ et $U$) on a $T \geq U$.
        \end{questionpartie}
    \end{partie}
    \begin{partie}[Successeur]
        \begin{questionpartie}
            Utilisez $i^\star_{T}$ pour déterminer le successeur dans $\mathcal{T}_n$ d'un élément $T \in \mathcal{T}_n$ (le plus petit tableau de $\mathcal{T}_n$ strictement plus grand que $T$).
        \end{questionpartie}
        \begin{questionpartie}
            Écrivez un algorithme de signature \verb|suivant(T: [entier]) → [entier]|  qui prend en entrée un élément $T$ de $\mathcal{T}_n$ et rend son successeur pour l'ordre lexicographique dans $\mathcal{T}_n$. Sa complexité devra être optimale (vous le justifierez).
        \end{questionpartie}
        \begin{questionpartie}
            En déduire un algorithme itératif dont vous donnerez la complexité permettant d'afficher à l'écran tous les éléments de $\mathcal{T}_n$.
        \end{questionpartie}
        \begin{questionpartie}
            Quelle serait la complexité de l'algorithme précédent si à la place d'afficher tous les éléments il rendait une liste contenant tous les éléments de $\mathcal{T}_n$ ?
        \end{questionpartie}

    \end{partie}
\end{exo}
\begin{exo}[Récursif]
    On va modifier un algorithme permettant de mélanger un tableau pour générer tous les éléments de $\mathcal{T}_n$. Soit l'algorithme suivant, que l'on doit à Fisher et Yates (1938) :
        \begin{tabbing}
            ccc\=cccc\=cccc\=cccc\=cccc\=cccc\=\kill
            \textbf{algorithme} \textsc{mélange}($T$){\bf :}\\
            \>\textbf{Pour chaque} $i$ de [0, n-2] {\bf :}\\
            \> \>\vline $\,$ $j$ ← un entier aléatoire de [i, n-1] \\
            \> \>\vline $\,$ $T[i]$, $T[j]$ ←$T[j]$, $T[i]$
        \end{tabbing}

    \begin{partie}[Aléatoire]
        \begin{questionpartie}
            Donnez la complexité de l'algorithme \verb|mélange(T:[entier])|.
        \end{questionpartie}
        \begin{questionpartie}
            Démontrez que si $T=[0, n-1]$, \verb|mélange(T)| va modifier $T$ en une permutation $T'$ qui peut être n'importe quel élément de $\mathcal{T}_n$.
        \end{questionpartie}
        \begin{questionpartie}
            Transformez l'algorithme \verb|mélange(T)| en un algorithme récursif de signature :\\ \verb|mélange_rec(T: [entier], i: entier)|, de telle sorte que \verb|mélange(T) = mélange_rec(T, 0)| (la variable interne $i$ de la boucle \verb|pour chaque| devient un paramètre de la fonction). 
        \end{questionpartie}
        \begin{questionpartie}
        {\em\bf (question optionnelle)} Montrez que la probabilité que $T$ soit modifié en $T'$ par \verb|mélange(T)| est la même pour tout élément $T'$ de $\mathcal{T}_n$
        \end{questionpartie}

        \end{partie}
        \begin{partie}[Déterministe]
        \begin{questionpartie}
            En déduire un algorithme récursif et {\bf sans tirage aléatoire} permettant d'afficher à l'écran tous les éléments de $\mathcal{T}_n$. Quelle est sa complexité ?
        \end{questionpartie}
        \begin{questionpartie}
            Explicitez les sorties à l'écran de votre algorithme Lorsqu'il affiche tous les éléments $\mathcal{T}_3$.
        \end{questionpartie}

    \end{partie}
\end{exo}

\begin{exo}[Optimal]
    Nous allons dans cette partie examiner un algorithme optimal que l'on doit à B. R. Heap (1963).
        \begin{tabbing}
            ccc\=cccc\=cccc\=cccc\=cccc\=cccc\=\kill
            \textbf{algorithme} \textsc{heap}($T$, $k$){\bf :}\\
            \>\textbf{si} $k = 1$ {\bf :}\\
            \> \>\vline $\,$ affiche $T$ à l'écran  \\
            \>\textbf{sinon} {\bf :}\\
            \> \>\vline $\,$ \textsc{heap}($T$, $k-1$)\\
            \> \textbf{Pour chaque} $i$ de [0, k-2] {\bf :}\\
            \> \>\vline $\,$\textbf{si} $k$ est pair {\bf :}\\
            \> \>\vline \>\vline $\,$ $T[i]$, $T[k-1]$ ←$T[k-1]$, $T[i]$\\
            \> \>\vline $\,$\textbf{sinon} {\bf :}\\
            \> \>\vline \>\vline $\,$ $T[0]$, $T[k-1]$ ←$T[k-1]$, $T[0]$\\
            \> \>\vline $\,$ \textsc{heap}($T$, $k-1$)\\
        \end{tabbing}

    \begin{partie}[Vérification]
        
        \begin{questionpartie}
            Explicitez les sorties à l'écran de l'exécution de $\textsc{heap}([0, 1, 2], 3)$.
        \end{questionpartie}
        \begin{questionpartie}
            Explicitez les sorties à l'écran de l'exécution de $\textsc{heap}([0, 1, 2, 3, 4], 3)$.
        \end{questionpartie}
    \end{partie}
    \begin{partie}[Propriétés]
        Soit $T$ un tableau. On va étudier ses modifications près exécution de l'algorithme. Pour cela, on note $T'$ le tableau $T$ après l'exécution de $\textsc{heap}(T, k)$.
            \begin{questionpartie}
                Démontrez que  $T[k:\;] = T'[k:\;]$
            \end{questionpartie}
        \begin{questionpartie}
                Démontrez que :
                \begin{itemize}
                    \item si $k$ est impair alors $T'[\;:k] = T[\;:k]$
                    \item si $k$ est pair alors $T'[\;:k]$ est la permutation circulaire de 1 élément du tableau $T[\;:k]$
            \end{itemize}
        \end{questionpartie}
        \begin{questionpartie}
            Démontrez que lors de l'appel de $\textsc{heap}(T, k)$, avec $k>1$, chaque élément de $T[\;:k]$ sera placé exactement une fois en position $T[k-1]$ lors des différents appels $\textsc{heap}(T, k-1)$
        \end{questionpartie}
        \begin{questionpartie}
            En déduire que l'algorithme $\textsc{heap}$ permet d'afficher à l'écran tous les éléments de $\mathcal{T}_n$.
        \end{questionpartie}
    \end{partie}
    \begin{partie}[Optimalité]
        \begin{questionpartie}
            Démontrez que lors de l'exécution de $\textsc{heap}(T, k)$, il y a eu exactement $k!$ échanges. En déduire la complexité de l'affichage à l'écran de tous les éléments de $\mathcal{T}_n$.
        \end{questionpartie}
        \begin{questionpartie}
            Proposez une version itérative de l'algorithme $\textsc{heap}$.
        \end{questionpartie}

    \end{partie}
\end{exo}

\end{document}