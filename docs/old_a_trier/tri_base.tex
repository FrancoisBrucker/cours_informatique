\todo[inline]{Valentin: j'ai arrêté ma relecture / refonte ici}

\clearpage
\begin{exo}[Tri : tri par base]

Le tri par base est une méthode de tri utilisant des nombres codés en binaire. 

\begin{partie}[Puissance]\label{puissance}

\begin{lstlisting}[language=Python]
def puissance(nombre, exposant):
    résultat = nombre
    compteur = exposant - 1
    while compteur > 0:
        résultat *= nombre
        compteur -= 1
    return résultat
\end{lstlisting}

\begin{questionpartie} 
Démontrez que la fonction \verb|puissance(x, y)| rend bien $x^y$.
\end{questionpartie}

\begin{questionpartie} 
Donnez la complexité la fonction \verb|puissance(x, y)|.
\end{questionpartie}

\begin{questionpartie} 
Donnez la complexité du {\bf problème} du calcul de $x^y$.
\end{questionpartie}
\end{partie}

\begin{partie}[Encodage binaire]

Tout entier $N>0$ peut s'écrire sous la forme :
$$
N = \sum_{i=0}^{n-1}a_i\cdot 2^i
$$
Avec :
\begin{itemize}
\item $a_i \in \{0, 1\}$ pour tout $0\leq i < n$
\item $a_{n-1} =1$
\end{itemize}

On peut alors ranger les coefficients $a_i$ dans une liste et associer à tout entier $N$ une liste $L_N$ de taille $n$ telle que :

\begin{itemize}
\item $L_N[i] \in \{0, 1\}$ pour tout $0\leq i < n$
\item $L_N[n-1] =1$
\item $N = \sum_{i=0}^{n-1}L_N[i]\cdot 2^i$
\end{itemize}

\begin{questionpartie} 
Exprimez la taille $n$ de $L_N$ en fonction de $N$.
\end{questionpartie}

\begin{questionpartie} 
Quel est l'entier associé à la liste $[0, 1]$?
\end{questionpartie}

\begin{questionpartie} 
Explicitez la liste $L_{11}$.
\end{questionpartie}
\todo[inline]{C'est un peu piégeux de choisir $11$, composé de 1, demander plutôt $23$ par exemple?}

Pour la suite, on aura besoin d'avoir des nombres encodés par des listes de même taille. On définit ainsi la liste $L_N^p$ de taille $p$, telle que :

\begin{itemize}
\item $L^p_N[i] \in \{0, 1\}$ pour tout $0\leq i < p$
\item $N = \sum_{i=0}^{p-1}L^p_N[i]\cdot 2^i$
\end{itemize}

\begin{questionpartie} 
Explicitez la liste $L_{11}^6$.
\end{questionpartie}

\begin{questionpartie} 
Donnez un algorithme prenant deux paramètres $q \geq p$ et $L_N^p$ et rendant $L_N^q$.
\end{questionpartie}

\begin{questionpartie} 
Donnez un algorithme prenant en paramètre $L_N^p$ et rendant $L_N$.
\end{questionpartie}
\end{partie}

\begin{partie}[Conversions]

On considérera dans la suite de cet exrcice que l'opérateur $//$ correspond à la division entière et l'opérateur $\%$ au modulo. On considérera que la complexité de ces deux opérateurs est $\mathcal{O}(1)$.


\paragraph{}Pour tout entier $p$ et $q$, la division euclidienne s'écrit ainsi $p = (p//q) \cdot q + (p \% q)$.

\begin{questionpartie} 
Donnez un algorithme permettant de rendre $L_N$ pour un entier $N$ passé en paramètre.
\end{questionpartie}

\begin{questionpartie} 
En utilisant la fonction \verb|puissance| de la partie~\ref{puissance} donner un algorithme permettant de rendre $N$ pour une liste $L_N$ passée en paramètre.
\end{questionpartie}

\begin{questionpartie} 
Optimisez l'algorithme de la question précédente pour qu'il soit de complexité linéaire (cet algorithme n'utilisera plus la fonction \verb|puissance|).
\end{questionpartie}
\end{partie}

\begin{partie}[Algorithme de tri]

On suppose pour cette question que l'on veut trier une liste $[L^p_{N_1}, \dots, L^p_{N_q}]$ selon la valeur des nombres $N_i$ (notez que la taille des listes est toujours la même). Par exemple 

\begin{itemize}
\item la liste :$[[0, 1, 0], [1, 0, 1], [1, 0, 0]]$ correspondant à $[L_2^3, L_5^3, L_1^3]$
\item  devra être triée en : $[[1, 0, 0], [0, 1, 0], [1, 0, 1]]$ correspondant à $[L_1^3, L_2^3, L_5^3]$.
\end{itemize}

L'algorithme ci-après est une façon de faire :

\begin{lstlisting}[language=Python]
def tri_base(liste_à_trier):
    p = len(liste_à_trier[0])

    for i in range(p):
        L0 = []
        L1 = []

        for L in liste_à_trier:
            if L[i] == 0:
                L0.append(L)
            else:
                L1.append(L)
        liste_à_trier = L0 + L1

    return liste_à_trier
\end{lstlisting}

\begin{questionpartie} 
Explicitez les étapes effectuées par l'exécution de \verb|tri_base([[0, 1, 0], [1, 0, 1], [1, 0, 0]])|.
\end{questionpartie}

\begin{questionpartie} 
Donner la complexité de la fonction \verb|tri_base|.
\end{questionpartie}

\begin{questionpartie} 
Démontrez que c'est bien un algorithme de tri.
\end{questionpartie}

\end{partie}

\begin{partie}[Conclusions]

Le tri par base peut très bien être une méthode de tri d'une liste de nombres.

\begin{questionpartie} 
Proposez une méthode utilisant le tri par base pour trier une liste de nombres entiers.
\end{questionpartie}

\begin{questionpartie} 
Donnez la complexité de cette méthode.
\end{questionpartie}

\begin{questionpartie} 
Comparez cette complexité avec la complexité du problème du tri.
\end{questionpartie}

\end{partie}
\end{exo}

\clearpage