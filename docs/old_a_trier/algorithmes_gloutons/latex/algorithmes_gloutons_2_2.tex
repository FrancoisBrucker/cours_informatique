% Options for packages loaded elsewhere
\PassOptionsToPackage{unicode}{hyperref}
\PassOptionsToPackage{hyphens}{url}
%
\documentclass[
]{article}
\usepackage{fullpage}
\usepackage{lmodern}
\usepackage{amssymb,amsmath}
\usepackage{ifxetex,ifluatex}
\ifnum 0\ifxetex 1\fi\ifluatex 1\fi=0 % if pdftex
  \usepackage[T1]{fontenc}
  \usepackage[utf8]{inputenc}
  \usepackage{textcomp} % provide euro and other symbols
\else % if luatex or xetex
  \usepackage{unicode-math}
  \defaultfontfeatures{Scale=MatchLowercase}
  \defaultfontfeatures[\rmfamily]{Ligatures=TeX,Scale=1}
\fi
% Use upquote if available, for straight quotes in verbatim environments
\IfFileExists{upquote.sty}{\usepackage{upquote}}{}
\IfFileExists{microtype.sty}{% use microtype if available
  \usepackage[]{microtype}
  \UseMicrotypeSet[protrusion]{basicmath} % disable protrusion for tt fonts
}{}
\makeatletter
\@ifundefined{KOMAClassName}{% if non-KOMA class
  \IfFileExists{parskip.sty}{%
    \usepackage{parskip}
  }{% else
    \setlength{\parindent}{0pt}
    \setlength{\parskip}{6pt plus 2pt minus 1pt}}
}{% if KOMA class
  \KOMAoptions{parskip=half}}
\makeatother
\usepackage{xcolor}
\IfFileExists{xurl.sty}{\usepackage{xurl}}{} % add URL line breaks if available
\IfFileExists{bookmark.sty}{\usepackage{bookmark}}{\usepackage{hyperref}}
\hypersetup{
  pdftitle={Algorithmes gloutons : comme heuristiques},
  hidelinks,
  pdfcreator={LaTeX via pandoc}}
\urlstyle{same} % disable monospaced font for URLs
\setlength{\emergencystretch}{3em} % prevent overfull lines
\providecommand{\tightlist}{%
  \setlength{\itemsep}{0pt}\setlength{\parskip}{0pt}}
\setcounter{secnumdepth}{-\maxdimen} % remove section numbering
\ifluatex
  \usepackage{selnolig}  % disable illegal ligatures
\fi

\title{Algorithmes gloutons : comme heuristiques}
\author{}
\date{}

\begin{document}
\maketitle

\hypertarget{but}{%
\subsection{But}\label{but}}

Montrer que, même s'ils ne réussissent pas toujorus à trouver la
solution optimale, les algorithmes gloutons sont souvent des
heuristiques bien pratiques.


\hypertarget{exercice-1-le-probluxe8me-du-voyageur-de-commerce}{%
\subsection{exercice 1 : le problème du voyageur de
commerce}\label{exercice-1-le-probluxe8me-du-voyageur-de-commerce}}

Le
\href{https://fr.wikipedia.org/wiki/Probl\%C3\%A8me_du_voyageur_de_commerce}{problème
du voyageur de commerce} peut s'énoncer comme suit :

étant donné un ensemble de villes reliées entre elles par des routes,
trouver l'itinéraire le plus court passant par chaque ville une et une
seule fois.

On suppose dans la suite de cet exercice que l'on connaît la distance
\(d(u, v)\) pour n'importe quel couple de villes \(u\) et \(v\).

\hypertarget{nombre-de-solutions}{%
\subsubsection{nombre de solutions}\label{nombre-de-solutions}}

combien de solutions possibles possède un problème du voyageur de
commerce à \(n\) villes ?

\hypertarget{algorithme-glouton}{%
\subsubsection{algorithme glouton}\label{algorithme-glouton}}

\begin{enumerate}
\def\labelenumi{\arabic{enumi}.}
\tightlist
\item
  Proposez un algorithme glouton qui résout ce problème.
\item
  Montrer que votre algorithme glouton n'est pas toujours optimal pour
  un ensemble de points du plan.
\end{enumerate}

\hypertarget{optimisation}{%
\subsubsection{optimisation}\label{optimisation}}

Proposez un moyen d'optimiser la solution obtenue par l'algorithme
glouton. On pourra par exemple remarquer que si l'on supprime 2 arêtes
disjointes d'un cycle on peut créer un autre cycle en ajoutant seulement
2 autres arêtes disjointes.

\hypertarget{exercice-2-coloration-de-graphes}{%
\subsection{exercice 2 : coloration de
graphes}\label{exercice-2-coloration-de-graphes}}

Le problème de la
\href{https://fr.wikipedia.org/wiki/Coloration_de_graphe}{coloration de
graphes} peut s'écrire comme suit :

Soit un graphe (simple) \(G = (V, E)\). Une fonction \(f\) de \(V\) dans
\(\\{1, \dots, \vert k \vert \\}\) (avec \(k \leq \vert V \vert\)) est
une \emph{coloration} de \(G\) si \(f(x) \neq f(y)\) pour toute arête
\(xy\) du graphe. Le nombre de couleurs différentes utilisées est \(k\).

Le \emph{nombre chromatique} d'un graphe \(G\), noté \(\chi(G)\) est le
nombre minimum de couleur qu'il faut pour le colorier.

Cette notion modélise très bien les problèmes de ressources partagées
(interférences entres antennes), d'incompatibilités ou encore de
coloration de cartes de géographie (ce que nous verrons).

\hypertarget{exemples}{%
\subsubsection{exemples}\label{exemples}}

Montrez que :

\begin{itemize}
\tightlist
\item
  les cycles paires sont 2 coloriables
\item
  les cycles impairs sont 3 coloriables et pas 2 coloriables
\item
  un graphe \(G=(V, E)\) est 2-coloriable si et seulement si il est
  \href{https://fr.wikipedia.org/wiki/Graphe_biparti}{\emph{bi-parti}}
  (si on eut partitionner \(V\) en deux ensembles \(V_1\) et \(V_2\) tel
  que chaque arête a un sommet dans \(V_1\) et un autre dans \(V_2\)).
\end{itemize}

\hypertarget{glouton}{%
\subsubsection{glouton}\label{glouton}}

Connaître le nombre chromatique d'un graphe est un problème NP-difficile
(un des problème les plus durs en informatique).

Proposez un algorithme glouton pour résoudre ce problème. Cet algorithme
doit prendre itérativement un sommet du graphe (dans un ordre
quelconque) et lui donner une couleur (qu'il ne remettra jamais en
cause).

Quel est sa complexité ?

Essayez sur le graphe suivant en prenant les sommet par ordre
alphabétique : {[}couleurs{]}(\{\{
``ressources/algorithmes\_gloutons/couleur\_graphe.png'' \}\})

\hypertarget{glouton-avec-ordre-choisi}{%
\subsubsection{glouton avec ordre
choisi}\label{glouton-avec-ordre-choisi}}

Quel ordre pensez-vous prendre ? Testez le sur le graphe précédent.

\hypertarget{prouvons-avec-glouton}{%
\subsubsection{prouvons avec glouton}\label{prouvons-avec-glouton}}

Montrez en utilisant l'algorithme que pour un graphe donné
\(\chi(G) \leq max_x(\delta(x)) +1\). Est-ce que cette borne est
atteinte ?

\hypertarget{exemple}{%
\subsubsection{exemple}\label{exemple}}

On reprend l'exemple des salles de cinéma de la première partie. Ecrivez
le problème du nombre minimum de salles de cinéma pour faire passer tous
les films sous forme d'un problème de coloration.

Application aux films :

\begin{itemize}
\tightlist
\item
  a : (2, 6)
\item
  b : (3, 8)
\item
  c : (5, 12)
\item
  d : (7, 9)
\item
  e : (11, 15)
\item
  f : (12, 17)
\item
  g : (13, 20)
\item
  h : (1, 4)
\end{itemize}

\hypertarget{exercice-3-cartes-de-guxe9ographie}{%
\subsection{exercice 3 : cartes de
géographie}\label{exercice-3-cartes-de-guxe9ographie}}

Le problème est, étant donné une carte de géographie, on veut la
colorier de telle sorte que chaque pays est une couleur différente de
ses voisins.

\hypertarget{coloration-de-cartes}{%
\subsubsection{coloration de cartes}\label{coloration-de-cartes}}

Montrez que l'on peut modéliser ce problème comme un problème de
coloration.

Et testez le sur l'exemple suivant : {[}carte{]}(\{\{
``ressources/algorithmes\_gloutons/carte.png'' \}\})

\hypertarget{planarituxe9}{%
\subsubsection{planarité}\label{planarituxe9}}

Un graphe est dit \emph{planaire} si on peut le représenter
graphiquement dans le plan sans arête qui se croise.

Montrez que le graphe de coloration d'une carte est planaire.

\hypertarget{formule-deuler}{%
\subsubsection{formule d'Euler}\label{formule-deuler}}

Soit \(G = (V, E)\) un graphe planaire (avec \(n = \vert V \vert\) et
\(m = \vert E \vert\)). On note \(F\) le nombre de face (le bord
extérieur étant considéré comme une face).

Montrez que : \(n - m + F = 2\) (cette relation est appelée
\emph{formule d'Euler}).

De là :

\begin{enumerate}
\def\labelenumi{\arabic{enumi}.}
\tightlist
\item
  en utilisant le fait que toute arête sépare exactement 2 faces et
  qu'une face a au moins 3 arêtes démontrez que \(m\leq 3n - 6\)
\item
  de l'inégalité ci-dessus, déduisez-en qu'il existe toujours un sommet
  de degré au plus 5
\item
  enfin, en utilisant l'algorithme de coloration prouver qu'il faut au
  maximum 6 couleurs pour coloriez toute carte de géographie
\end{enumerate}

\hypertarget{exercice-4-le-probluxe8me-du-sac-uxe0-dos}{%
\subsection{exercice 4 : le problème du sac à
dos}\label{exercice-4-le-probluxe8me-du-sac-uxe0-dos}}

Le
\href{https://fr.wikipedia.org/wiki/Probl\%C3\%A8me_du_sac_\%C3\%A0_dos}{problème
du sac à dos} est un exemple de problème d'optimisation. Il fait parti
des problèmes les plus durs du monde car les solutions n'entretiennent
pas de relations les unes avec les autres, il faut a priori toutes les
regarder pour trouver la meilleure, et il y en a beaucoup.

Il est possible de modéliser beaucoup de problèmes courants par un
problème de sac à dos, en particuliers les:

\begin{itemize}
\tightlist
\item
  problèmes de découpe pour minimiser les chutes
\item
  problèmes de remplissage (déménagement)
\end{itemize}

\hypertarget{uxe9noncuxe9-du-probluxe8me}{%
\subsubsection{énoncé du problème}\label{uxe9noncuxe9-du-probluxe8me}}

On dispose de :

\begin{itemize}
\tightlist
\item
  \(n\) objets ayant chacun un poids \(w_i\) (\emph{weight}) et une
  valeur nutritionnelle \(p_i\) (\(1 \leq i \leq n\))
\item
  d'un sac à dos d'une contenance de \(W\)
\end{itemize}

On veut maximiser la valeur nutritionnelle que l'on peut emporter avec
notre sac.

\hypertarget{on-essaie}{%
\subsubsection{on essaie}\label{on-essaie}}

On suppose que l'on est un randonneur et que l'on peut emporter 3
produits :

\begin{itemize}
\tightlist
\item
  produit A, 2kg, 100kcal
\item
  produit B, 2kg, 10kcal
\item
  produit C, 3kg, 120kcal
\end{itemize}

Selon la valeur du sac à dos quel est la quantité maximale d'énergie que
le randonneur peut emporter avec lui ?

\hypertarget{sac-uxe0-doc-fractionnel}{%
\subsubsection{sac à doc fractionnel}\label{sac-uxe0-doc-fractionnel}}

Si l'on peut prendre qu'une partie des objets (comme pour une poudre ou
un liquide), le problème peut être résolu par un algorithme glouton.

Problème :

\begin{itemize}
\tightlist
\item
  entrée :

  \begin{itemize}
  \tightlist
  \item
    liste de produits décrit par leur masse et le prix total
  \item
    une masse totale transportable
  \end{itemize}
\item
  sortie : une liste de produits et leur masse qui maximise l'énergie
  pour une masse ne dépassant pas la masse transportable
\end{itemize}

\hypertarget{ruxe9solvez-le-probluxe8me-avec-les-donnuxe9es-pruxe9cuxe9dentes}{%
\paragraph{résolvez le problème avec les données
précédentes}\label{ruxe9solvez-le-probluxe8me-avec-les-donnuxe9es-pruxe9cuxe9dentes}}

On suppose que l'on peut découper les objet pour obtenir une fraction de
leurs valeurs. résolvez le problème pour une capacité de sac de 4kg et
de 5kg.

\hypertarget{algorithme-glouton-1}{%
\paragraph{algorithme glouton}\label{algorithme-glouton-1}}

Proposez un algorithme glouton pour résoudre ce problème et montrer
qu'il est optimal.

\hypertarget{sac-uxe0-dos-non-fractionnel}{%
\subsubsection{sac à dos non
fractionnel}\label{sac-uxe0-dos-non-fractionnel}}

\hypertarget{si-on-ne-peut-pas-couper}{%
\paragraph{si on ne peut pas couper ?}\label{si-on-ne-peut-pas-couper}}

Donner un exemple où l'algorithme glouton ne donne pas la solution
optimale si l'on ne peut pas prendre une partie fractionnelle d'un
produit.

\hypertarget{solution-optimale}{%
\paragraph{solution optimale}\label{solution-optimale}}

On peut trouver un algorithme optimal pour le problème du sac à dos en
remarquant que l'on peut construire une solution optimale avec \(i\)
objets à partir de solutions optimales à \(i-1\) objets.

En effet la solution optimale à \(i\) objets pour une capacité \(W\) est
soit :

\begin{itemize}
\tightlist
\item
  une solution optimale à \(i-1\) objets pour une capacité \(W\) si on
  ne prend pas l'objet \(i\),
\item
  une solution optimale à \(i-1\) objets pour une capacité \(W - w_i\)
  si on prend l'objet \(i\).
\end{itemize}

\hypertarget{algorithme-optimal}{%
\paragraph{algorithme optimal}\label{algorithme-optimal}}

Ecrivez l'algorithme permettant de résoudre le problème.

Et explicitez pourquoi cet algorithme n'est pas glouton.

\hypertarget{complexituxe9-de-lalgorithme}{%
\paragraph{Complexité de
l'algorithme}\label{complexituxe9-de-lalgorithme}}

Quel est la complexité de cet algorithme.

\hypertarget{le-million-de-dollars}{%
\paragraph{Le million de dollars}\label{le-million-de-dollars}}

Le problème du sac à dos fait partie des problème ``les plus durs de
l'informatique'', or on a un algorithme polynomial pour le
résoudre\ldots{} Soit on vient de démontrer
\href{https://fr.wikipedia.org/wiki/Probl\%C3\%A8mes_du_prix_du_mill\%C3\%A9naire\#Probl\%C3\%A8me_ouvert_P_=_NP}{P
= NP} et on a gagné 1millions de dollars, soit il y a un piège.

Quel est ce piège ?

\end{document}
